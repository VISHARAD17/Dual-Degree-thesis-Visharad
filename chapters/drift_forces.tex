\chapter{DRIFT FORCES}
\section{Introduction}
Mean drift forces are the mean non-linear forces which can be obtained using perturbation theory. 
Forces explained and computed in the previous chapters
are first order forces. While solving boundary value problems to get the potential, pressure, forces, 
motions, etc. usually
second order terms are ignored in order to reduce the complexity of the problem and also the mangitude of
second order terms which are these mean forces is also very small. But when the actual responses are very high,
these second order mean forces become important. These forces are proportional to swuare of the wave height 
while the 
first order are proportional to the wave height. Primarily, mean drift forces are computed using two menthods:
(a) far feild method and (b) near field method.

This chapter deals with the mean drift force and moments acting on a stationary vessel in regular waves. This chapter
presents the complete derivation of the second order force on an arbitrary shaped body moving with a steady 
forward speed in regular waves using 3D panel based potential theory. Far field method introduced by
\cite{maruo1957excess} is based on the diffracted and radiated wave energy and momentum flux at infinity. Near feild
method introduced by \cite{boese1970einfache} uses direct hydrodynamic pressure integration over the wetted surface and 
is based on 3D panel based potential theory.

\section{Pertubation Expansions}
\label{sec:perturbation_exp}
Perturbation theory is used to obtain the approximation solution upto a certain accuracy. Solution or quantity is 
expressed as a power series using a small paramater known as perturbation parameter. perturbation expression
starts with an avarage solution.
For example, 
\begin{equation}
    A = A_0 + \epsilon A_1 + \epsilon^2 A_2 + \cdots
\end{equation}
where, $\epsilon$ is the perturbation paramter and $A_0$ is the avarage solution of $A$.

Now, Assuming small amplitute motion oscillations about mean position of the body, approximations can be obtain 
up to second order with respect to the wave amplitute. Perturbation expressions of quantities 
of interest using a small parameter $\epsilon$ of the order of the wave slope, are given below :

\begin{itemize}
    \item [1.] Expressions for velocity potential $\left(\phi = \phi_T\,e^{i\,w_e\,t}\right)$ :
    \begin{equation}
        \phi = \epsilon \phi^{(1)} + \epsilon^2 \phi^{(2)} + \cdots
    \end{equation}

    \item [2.] The free surface elevation $(\zeta)$ :
    \begin{equation}
        \zeta = \epsilon \zeta^{(1)} + \epsilon^2 \zeta^{(2)} + \cdots
    \end{equation}

    \item [3.] The relative wave elevation:
    \begin{equation}
        \zeta_r = \epsilon \zeta_r^{(1)} + \epsilon^2 \zeta_r^{(2)} + \cdots
    \end{equation}

    \item [4.] The vessel motions :
    \begin{equation}
        \vec{\eta}(t) = \epsilon \vec{\eta}^{(1)} + \epsilon^2 \vec{\eta}^{(1)} + \cdots
    \end{equation}
    where, $\vec{\eta}^{(1)} = (\eta_1, \eta_2, \eta_3, \eta_4, \eta_5, \eta_6)$ represents the first order surge, 
    sway, heave, roll, pitch and yaw motions respectively.

    \item [5.] The pressure field in the fluid :
    \begin{equation}
        \label{eq:per_perssure}
        p = p^{(0)} + \epsilon p^{(1)} + \epsilon^{2} {^{(2)}} + \cdots
    \end{equation}
\end{itemize}



\section{Pressure and Force derivations}


\subsection{Pressure derivation}
The pressure using Bernoullis equation is given as:
\begin{equation}
    p = \frac{1}{2}\rho\,U^2 - \rho\,\frac{\partial \phi_T}{\partial t} - \frac{1}{2}\rho |\nabla\phi_T|-\rho\,g\,z^2
\end{equation} 
After substituting the above expression in eq.(\ref{eq:per_perssure}), pressure quatities with respect 
to different orders of $\epsilon$ are given as 
\begin{equation}
    p^{(0)} = - \rho g(z_B + z_0)
\end{equation}
where, $(\vec{X_0}) = (X_0, Y_0, Z_0)$ is the location of body coordinate system origin with respect to the global 
coordinate system and $\vec{x_B}=(x_B, y_B, z_B)$ the location of the point where the quantiy is begin computed in the body coordinate
system.
First order expression for pressure is given as:
\begin{align}
    P^{(2)} &= -\rho\frac{\partial \phi^{(2)}}{\partial t} + \rho\,U\frac{\partial \phi^{(2)}}{\partial x} -\frac{\rho}{2}
    \left\{\left(\frac{\partial \phi^{(1)}}{\partial x}\right)^2 + \left(\frac{\partial \phi^{(1)}}{\partial y}\right)^2 + 
    \left(\frac{\partial \phi^{(1)}}{\partial z}\right)^2\right\} \\ \nonumber
    &-\rho\left\{x^{(1)}\cdot\nabla \left(\frac{\partial \phi^{(1)}}{\partial t} - U\frac{\partial \phi}{\partial x}\right)\right\}
    - \rho g z^{(2)}
\end{align}
where, $z^{(2)} = [\eta_4\eta_6 x_B + \eta_5\eta_6 y_B - \frac{1}{2}(\eta_4^2 + \eta_5^2)z_B]$. All the potentails and 
their derivatives are calculated at mean position.
\subsection{Relative wave elevation}
Relative wave elevation is the distance between the wave surface and the instantaneous waterline. To calculate the
relative wave elevation, first absolute wave elevation needs to be calculated along the waterline. The waterline is obtained 
by extracting the edges of the panels which are at the water surface. According to the dynamic free surface boundary 
condition the pressure at free surface is zero ( guage pressure ) i.e. at $\zeta=0$, $p=0$. Hence, 
\begin{equation}
    \rho g \zeta + \rho\frac{\partial \phi}{\partial t} - \rho U \frac{\partial \phi}{\partial x} =0 \quad \text{on} \quad z=\zeta 
\end{equation}
Therefore the above equation yeilds :
\begin{equation}
    \zeta^{(1)} = -\frac{1}{g}\left(i\omega_e \phi_T - U\frac{\partial \phi_T}{\partial x}\right) e^{i\,\omega_e\,t}
\end{equation}
The relative wave elevation is  then obtained by subtracting the total movement of the body in $z$-direction from the
absolute wave elevation.
\begin{equation}
    \zeta^{(1)}_r =\zeta^{(1)} - (\eta_3 - \eta_5x + \eta_4y)
\end{equation}
where, $(x, y)$ are the centroid of the waterline panels or the waterline element.


\subsection{Forces and moments}
The hydrodynamic force is given as 
\begin{equation}
    F_H = -\int_S P\,n_j\,ds \quad \quad j=1, 2,\cdots, 6
\end{equation}
Using the perturbation expansions, equations for force can be written as
\begin{equation}
    \label{eq:per_force}
    \vec{F} = -\left(\int_{S_0}ds+\int_{wl}\zeta_r dl\right)(p^{(0)} + \epsilon p^{(1)} +
    \epsilon^2 p^{(2)}) (\vec{n}^{(0)}+\epsilon(\vec{\theta}^{(0)} \times \vec{n}^{(0)})+\epsilon^2H\vec{n}^{(0)})
\end{equation} 
The waterline integral arises due to the condideration of the first order wetted surface 
area $S_1$ which is the additional instantaneous surface of the hull below water considering 
both wave elevation and the motion of the body.
\begin{equation}
    \int_{S_1}\cdots ds = \int_{wl} dl \int_{0}^{\zeta_r}\cdots \frac{dz}{\sqrt{1-n^2_3}}
\end{equation}

Here, $wl$ is the waterline of the ship, $\zeta_r$ is the relative wave elevation and $dz/\sqrt{1-n^2_3}$ is the inclined height for 
non-wall sided surface. Now, after expanding the above equation and separating the terms with 
$\epsilon$, $\epsilon^1$ and $\epsilon^2$ gives the zeroth, first and second order force respectively.
Now, after substituting relative wave elevation in the eq(\ref{eq:per_force}) and extracting terms corresponding
to $\epsilon^2$, second order force can be obtained whose equation is given below :
\begin{align}     
    F &= -\int_{wl}\frac{1}{2}\rho g (\zeta_r^{(1)})^2 \frac{\vec{\eta}^{(0)}}{\sqrt{1-\eta_3^2}} dl + 
    \int_{S_0}\rho\left(\frac{\partial \phi^{(2)}}{\partial t} - U\frac{\partial \phi^{(2)}}{\partial x}\right)\vec{\eta}^{(0)}ds \\ \nonumber
    &+ \int_{S_0} \frac{\rho}{2}\left\{\left(\frac{\partial \phi^{(1)}}{\partial x}\right)^2 + \left(\frac{\partial \phi^{(1)}}{\partial x}\right)^2
    + \left(\frac{\partial \phi^{(1)}}{\partial x}\right)^2\right\}\vec{\eta}^{(0)} \,ds \\ \nonumber
    &+\int_{S_0}i\omega_e\rho\left\{(\eta_1-\eta_6y_B+\eta_5z_B)\frac{\partial \phi^{(1)}}{\partial x} \right. \\ \nonumber
    &\left. +(\eta_2+\eta_6x_B-\eta_4z_B)\frac{\partial \phi^{(1)}}{\partial y} + (\eta_3-\eta_5x_B+\eta_4y_B)
    \frac{\partial \phi^{(1)}}{\partial z}\right\}\vec{\eta}^{(0)} \,ds \\ \nonumber
    &- \rho g A^{(0)}\left[\eta_4 \eta_6 x_{B,f} + \eta_5\eta_6 y_{B,f} + \frac{1}{2}(\eta_4^2+\eta_5^2)Z_0\right]\hat{k} \\ \nonumber
    &-\omega^2_e\{-\eta_2\eta_6m + \eta_4\eta_6m z_g -\eta_6\eta_6 m x_g + \eta_3 \eta_5 m + \eta_4\eta_5y_g
    - \eta_5\eta_5 m x_g\} \hat{i} \\ \nonumber
    &-\omega^2_e\{\eta_1\eta_6m + \eta_5\eta_6m z_g -\eta_6\eta_6 m y_g - \eta_3 \eta_4 m + \eta_4 \eta_4 m y_g
    + \eta_4 \eta_5 m x_g\} \hat{j} \\ \nonumber
    &-\omega^2_e\{-\eta_1\eta_5 m - \eta_5\eta_5m z_g +\eta_5\eta_6 m y_g + \eta_2 \eta_4 m - \eta_4 \eta_4 m z_g
    + \eta_4 \eta_6 m x_g\} \hat{k}
\end{align}

Similarly, the second order moment can be given as :
\begin{align}
    \vec{M}^{(2)} &= -\int_{wl}\frac{1}{2}\rho g (\zeta_r^{(1)})^2 (x_B\times \vec{n}^{(0)}) dl \\ \nonumber
    &+\rho\int_{S_0}\left(\frac{\partial \phi^{(2)}}{\partial t} - U\frac{\partial \phi^{(2)}}{\partial x}\right)
    (\vec{x_B}\times\vec{n}^{(0)})\,ds \\ \nonumber
    &+\frac{\rho}{2}\int_{S_0}\left\{\left(\frac{\partial \phi^{(1)}}{\partial x}\right)^2
    \left(\frac{\partial \phi^{(1)}}{\partial y}\right)^2 + \left(\frac{\partial \phi^{(1)}}{\partial z}\right)^2
    \right\}(\vec{x_B}\times\vec{n}^{(0)})\,ds \\ \nonumber
    &+i\omega_e\rho\int_{S_0}
    \left\{(\eta_1-\eta_6y_B+\eta_5z_B)\frac{\phi^{(1)}}{\partial x} \right. \\ \nonumber 
    &\left. + (\eta_2-\eta_6x_B-\eta_4z_B)\frac{\phi^{(1)}}{\partial y}\right. \\ \nonumber
    &\left. + (\eta_3-\eta_5x_B+\eta_4y_B)\frac{\phi^{(1)}}{\partial z}\right\}
    (\vec{x_B}\times \vec{n}^{(0)})\,ds \\ \nonumber
    &-\omega_e^2\{\eta_2\eta_4my_g-\eta_1\eta_6mz_g-\eta_4\eta_6I_{54}
    -eta_5\eta_6I_{55}-\eta_6\eta_6I_{56} \\ \nonumber
    &+\eta_3\eta_4mz_g-\eta_1\eta_5my_g+\eta_4\eta_5I_{64}+\eta_5\eta_5 I_{65}
    +\eta_5\eta_6 I_{66} \}\hat{i} \\ \nonumber
    &-\omega_e^2\{\eta_3\eta_5mz_g-\eta_2\eta_6mz_g+\eta_4\eta_6I_{44}+\eta_5\eta_6 I_{45}
    +\eta_6\eta_6I_{46}\\ \nonumber
    &+\eta_1\eta_5mx_g-\eta_2\eta_4mx_g-\eta_4\eta_4 I_{64}-\eta_4\eta_5 I_{65}-\eta_4\eta_6 I_{66}\} \hat{j} \\ \nonumber
    &-\omega_e^2\{\eta_1\eta_6mx_g-\eta_3\eta_5my_g-\eta_4\eta_5my_g-\eta_4\eta_5 I_{44}
    -\eta_5\eta_5 I_{45}-\eta_5\eta_6 I_{46} \\ \nonumber
    &+\eta_2\eta_6my_g-\eta_3\eta_4mx_g+\eta_4\eta_4 I_{54}+\eta_4\eta_4 I_{55}
    +\eta_4\eta_6 I_{56} \}\hat{k} \\ \nonumber
    &+\rho g \left[-V^{(0)}\eta_1\eta_6+V^{(0)}\eta_4\eta_5x_{CB}-V^{(0)}\eta_5\eta_6z_{CB}
    -\frac{1}{2}V^{(0)}(\eta_4^2-\eta_6^2)y_{CB} \right. \\ \nonumber
    &\left.-\eta_4\eta_6L_{12} -\eta_5\eta_6L^{22}-\frac{1}{2}(\eta_4^2+\eta_5^2)
    Z_0A^{(0)}y_f+\eta_5\eta_6V^{(0)}Z^{(0)}\right]\hat{i} \\ \nonumber
    &+\rho g \left[-V^{(0)}\eta_2\eta_6+V^{(0)}\eta_4\eta_6 z_{CB}+\frac{1}{2}V^{(0)} 
    (n\eta-5^2-\eta_6^2)x_{CB} \right. \\ \nonumber
    &\left. +\eta_4\eta_6 L_{11}+\eta_5\eta_6L_{12}+\frac{1}{2}(\eta_4^2+\eta_5^2)Z_0A^{(0)}x_f
    -\eta_4\eta_6 Z_0V^{(0)}\right]\hat{j} \\ \nonumber
    &+\rho g V^{(0)}(\eta_1\eta_4+\eta_2\eta_5+\eta_5\eta_6x_{CB}-\eta_4\eta_6y_{CB})\hat{k}
\end{align}

added mass 
RAOS - random angles - 90-270 and one angle all 6
added mass , radiation damping, exciting force - one angle all 6
