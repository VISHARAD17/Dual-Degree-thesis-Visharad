\chapter{MATHEMATICAL MODEL}
In this thesis, a ship moving with a steady speed of $U$ in deep water with regular 
waves of wave amplitude $A$ and incident frequency $w_I$ traveling at an angle 
$\beta$ relative to the surge direction of the ship is considered. To address the seakeeping
problem, potential flow techniques are commonly employed. With the advancements in computing power, 
it is now possible to use the three-dimensional panel based potential theory method to compute 
the wave load.

\section{Co-ordinate System}

In order to set up the mathematical model, two coordinate systems are defined; one of them is the 
Global coordinate system ( GCS ), whose origin is located at a calm water level. The other one is 
the Body seakeeping coordinate system ( BCS ), whose origin is located at the midship on the 
intersection of the water line and centerline of the ship. It is further assumed that the $x$-axis 
of the GCS points towards the east direction, whereas the $x$-axis of BCS points towards the sway 
direction of the ship. Coordinates of points represented in GCS are expressed as 
$\boldsymbol{x^e} = (x^e, y^e, z^e)$ wheres for the points in BCS it is expressed as 
$\boldsymbol{x^s} = (x^s, y^s, z^s)$. Global coordinate system, i.e., GCS, 
denotes the inertial frame of reference. Body geometry and body parameters, 
such as the location of the vertical center of gravity and radii of gyration, 
are defined with respect to BSC.

The translating frame of reference, BCS allows the formulation of the vessel response in 
six degrees of freedom due to incident waves and steady current of speed $U$ in $-x$ direction 
which is equivalent to forward speed with respect to GCS. By assuming a small wave amplitude, 
a linear boundary value problem can be established to determine the velocity potential.


\section{Velocity potential}
Assuming that the fluid flow is inviscid, incompressible, and irrotational the total velocity potential at any point inside the fluid domain is given as :
\begin{equation}
    \boldsymbol{\Phi} (\vec{x}, t) = [\, -Ux + \phi_P(\vec{x})\,] + [\, \phi_I(\vec{x}, \beta, \omega_I) + \phi_S(\vec{x}, \beta, \omega_I) + \sum_{j=1}^{6}n_j\phi_j(\vec{x}, U, \omega_e) \,]\, e^{i w_e t}
\end{equation}
where, 
\begin{itemize}
    \item $\omega_I$ is the incident wave frequency.
    \item $\omega_e$ is the encountering wave frequency.
    \item $\phi_P(\vec{x})$ is the potential due to the perturbation of steady translation of stream.
    \item $\phi_I(\vec{x}, \beta, \omega_I)$ is the incident potential due to incident waves.
    \item $\phi_S(\vec{x}, \beta, \omega_I)$ is the scattering potential due to the reflected 
    waves from the body.
    \item $\phi_j(\vec{x}, U, \omega_e)$ is the radiation potential generated due to the
    motion of body in $j^{th}$ mode of motion.
\end{itemize}
In the above equation, the perturbation potential $\phi_P$ has a relatively insignificant 
effect on the total 
potential at low to moderate ship speed and so it is ignored to reduce the complexity 
of the problem. Hence, the final equation for velocity potential is given as :
\begin{equation}
    \label{eq:velocity_potential}
    \boldsymbol{\Phi} (\vec{x}, t) = -Ux + [\, \phi_I(\vec{x}, \beta, \omega_I) + \phi_S(\vec{x}, 
    \beta, \omega_I) + \sum_{j=1}^{6}n_j\phi_j(\vec{x}, U, \omega_e) \,]\, e^{i w_e t}
\end{equation}
In the above equation, $\omega_I$ is the 
frequency of the wave experienced by the vessel when it is stationary. According,
to the Doppler effect the actual frequency experienced by a body with stedy speed is different 
than the incident frequency of the source. Hence, in this problem, $\omega_e$ is the frequency of 
the wave experienced by the vessel when it is moving with a 
certain steady speed $U$, which is expressed in terms of incident or normal wave 
frequency $\omega_I$ , forward speed $U$, and incident angle $\beta$ as :
\begin{equation}
    \label{eq:omega}
    \omega_e = \omega_I - \frac{\omega_I^2}{g}\,U\cos\beta
\end{equation}

%%%%%%%%%%%%%%%%%%%%%%%%%%%%%%% Governing equation %%%%%%%%%%%%%%%%%%%%%%%%%%%%%%%%%%%%%%%%%%%%%%
\section{Governing equation}
The governing equations of the fluid flow are the continuity equation ( conservation of mass ) 
and the Navier-stokes equation ( conservation of momentum ). Under the assumption of inviscid 
and irrotational flow, the Navier-stokers equations reduce to Laplace equation given as :
\begin{equation}
    \label{eq:laplace_eq}
    \nabla^2 \boldsymbol{\Phi} (\vec{x}, t) = 0
\end{equation}

\section{Boundary Conditions}
\label{sec:Boundary condition}
In addition to the governing equation, the velocity potential satisfies the boundary 
conditions over the fluid boundaries. The total boundary surface $S$ comprises the 
free surface $S_F$, mean body surface $S_B$, bottom surface $S_Z$, and the radiation surface
 $S_\infty$ bounding the horizontal infinite fluid domain. The surfaces are shown in 
 Fig. \ref{fig:boundary_surfaces} below.
\begin{figure}[H]
	\centering
	\includegraphics[width = 0.65\textwidth]{photos/boundary_surfaces.png}
	\caption{Fluid boundary surfaces}
	\label{fig:boundary_surfaces}
\end{figure}
The boundary conditions to be satisfied by the potential functions are shown below:

\begin{itemize}
    \item[1.] \underline{Kinematic free surface boundary condition} : 
    The velocity of the fluid in the direction normal to the free surface is equal to the velocity of the free surface. If the free surface is given by $z = \eta(t, x, y)$, then the boundary condition is given by
    \begin{equation}
        \label{eq:kin_free_surface_cond}
        \frac{\partial \eta}{\partial t} = \frac{\partial \phi}{\partial z} - \frac{\partial \phi}{\partial x} \frac{\partial \eta}{\partial x} - \frac{\partial \phi}{\partial y}\frac{\partial \eta}{\partial y} \quad \text{over} \; z
    \end{equation}
    
    \item[2.] \underline{Dynamic free surface boundary condition} : 
    The pressure on the free surface is constand and is obtained by the application of Bernoulli's 
    equation over the free surface.
    \begin{equation}
        \label{eq:dyn_free_surface_cond}
        \left[\left(i\omega_e - u\frac{\partial}{\partial x}\right) + g\frac{\partial}{\partial z}\right](\phi_I, \phi_D, \phi_j) = 0 \quad \text{on} \; z = 0
    \end{equation}
    
    \item[3.] \underline{Radiation boundary condition} :
    The waves generated by the oscillating body propagate outward from the body to $\infty$ in the fluid domain, unbounded horizontally. This boundary condition is also referred to as the Sommerfeld radiation condition.
    \begin{equation}
        \label{eq:sommerfel_rad_cond}
        \lim_{kr\rightarrow \infty}\sqrt{kr}\left(\frac{\partial}{\partial r} -jk\right)(\phi_j - \phi_I) = 0 \quad \text{for}\; i = 1, 2, \cdots 6
    \end{equation}
    
    % \newpage
    \item[4.] \underline{Bottom Boundary condition} :
    The normal velocity of the fluid at the bottom boundary is equal to the normal velocity of the boundary. For an impenetrable seabed in deep water, the normal velocity is zero, and the boundary condition is given by 
    \begin{equation}
        \frac{\partial \phi}{\partial z} = 0 \quad \text{over the bottom}\; z = -\infty
    \end{equation}
    \item[5.] \underline{Body surface boundary condition}:
    The velocity of the fluid in the direction normal to the body boundary is equal to the normal velocity of the body boundary over the instantaneous underwater surface $S$.
    \begin{align}
        \label{eq:body_surface_boundary_cond}
        &\frac{\partial \phi_j}{\partial n} = i\omega_e n_j + Um_j \quad \text{on}\; S \\
        \label{eq:radiation_boundary}
        &\frac{\partial \phi_I}{\partial n} + \frac{\partial \phi_D}{\partial n} = 0 \quad \text{on}\; S
    \end{align}
\end{itemize}
\newpage
Here, $\vec{n} = (n_1, n_2, n_3)$ is the unit normal pointing outward from 
the hull surface and $(n_4, n_5, n_6)=\vec{r}\times \vec{n}$, where, $r$ 
is the position vector of a point on the surface. The linear incident wave 
potential satisfying the above boundary conditions is given by:
\begin{equation}
    \phi_I = \frac{igA}{\omega_I} e^{-ik_I(x\cos \beta + y\sin \beta)}e^{kz}
\end{equation}

 Here, $\omega_I$ represents the incident wave frequency, not the encounter frequency.
This forward speed boundary value problem can be solved using potential theory using 
infinite depth free surface green function. In general, free surface and body boundary conditions 
are non-linear and hence an exact solution is not possible. However, since the governing equation
is linear its solutions can be obtained by using perturbation approach.