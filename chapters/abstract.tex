\abstract
Anticipating the ship's movements is a critical aspect that needs to be considered both in 
the initial design phase and later during the ship's service life. It is essential to 
comprehend the motion characteristics of the vessel in all six degrees of freedom to 
ensure safe operation in challenging marine environments.With the increasing number of large ships,
there is a need to anticipate their optimal performance concerning travel time, fuel efficiency, 
as well as ensuring the security of cargo and personnel along specified shipping routes. 
This can be accomplished by employing user-friendly and effective numerical tools that can 
predict the hydrodynamic loads acting on floating vessels when moving at a constant forward 
speed and calculate their motion responses in diverse wave conditions as per the requirenments. 
The current dissertation outlines the integration of a three-dimensional potential theory-based 
approach, developed in Python, which takes into account the effects of forward speed. 
This paper elucidates the theoretical formulation, numerical implementation, as well as the 
comparisons with the outcomes of different programs. Additionally, the thesis discusses the 
near-field approach employed to calculate the mean drift force.

As a part of this dual degree project, a method to compute the wave forces and motions with 
the forward speed effect has been added into the web-based analysis tool {\bf HydRA} developed by 
the students of MAV lab students of IIT Madras. KCS and KVLCC2 ship hull forms are analyzed 
and validated against the MDLHydro software developed by Dr. Amitava Guha. 
The comparison results are found to be in excellent agreement.

\noindent {\bf \large Keywords} : Potential theory, laplace equation, boundary element method ( BEM ), 
Hydrodynamic, Wave structure interaction, multibody interaction, forward speed, mean drift force.