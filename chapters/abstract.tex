\abstract
Prediction of the dynamic motions of the ship is an essential aspect in the early stages of design as well as later during the service life of the ship. It is crucial to know the motion characteristics of the ship along the six degrees of freedom to operate it safely in adverse sea conditions. The growing number of large ships demands predicting their optimum performance with respect to travel time, fuel efficiency, and the safety of cargo and personnel in specified shipping routes. This can be achieved using efficient and easy-to-use numerical tools capable of predicting hydrodynamics loads on floating vessels with steady forward speed and solving their motion responses in various wave conditions. This project explains the implementation of a three-dimensional potential theory-based program in Python with the consideration of forward speed effects. The theoretical formulation, numerical implementation, and result comparisons are presented in this thesis.

As a part of this dual degree project, a frequency domain 3D panel-based program has been developed and extensively compared against the outputs of MDLHydroD software. KCS and KVLCC2 ship hull forms are analyzed and validated against the MDLHydro software developed by Dr. Amitava Guha. The comparison results are found to be in excellent agreement.

    
\noindent {\bf \large Keywords} : Potential theory, Laplace equation, Boundary element method ( BEM ), Hydrodynamic, Wave structure interaction, Multibody interaction, forward speed, drift force.