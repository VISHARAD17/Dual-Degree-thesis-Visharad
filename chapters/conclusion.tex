\chapter{CONCLUSION}

% A new numerical web-based frequency domain tool is developed to compute hydrodynamic force 
% and motions with low to moderate forward speed. The effect of forward speed was implemented by 
% changing normal incident frequencies with the encounter frequencies as incident frequencies
% changes as the object moves according to the Doppler effect. Also, the boundary conditions were also
% changes where forward speed terms and encountering frequencies were used to define them. Thus, the 
% results obtained then were compared against MDLHydroD softwares results. For comparison KCS and KVLCC2
% vessels' hull structures were used. Results show good agreement with the MDLHydroD software.
% Drift force are computed using the perturbation theory. All the terms of interest for drift force are 
% written using a small perturbation parameter till the second degree. After, substituting these terms 
% in the integral equations of force and moment. The terms corresponding to second degree of perturbation 
% parameter are taken to get the second order force and moment. Comparisons of drift force is done 
% between MDLHydroD and WAMIT6. Since, MDLHydroD also implements the effect of hull emergence angle, hence
% the value of drift force is less for some mode of motions as compared to WAMIT6. 


A new numerical tool is developed using a web-based frequency domain approach 
for calculating hydrodynamic forces and motions at low to moderate forward speeds.
To account for the effect of forward speed, incident frequencies were adjusted to 
encounter frequencies using the Doppler effect. Additionally, boundary conditions are 
modified by incorporating forward speed $U$ and encounter frequencies. The accuracy 
of the proposed method was verified by comparing the results against those obtained 
from the MDLHydroD software for the KCS and KVLCC2 vessel hull structures. While programming,
the code is implemented in such a way that the simulation can be performed for both 
single body case and multiple body case, which is an advantage over MDLHydroD.
The computed drift forces are determined using perturbation theory, and the second-order 
force and moment terms were obtained by considering perturbation parameter terms up to the 
second degree. A comparison of the drift force between MDLHydroD and WAMIT6 is also performed, 
where it is found that the value of drift force is lower for some modes of motion in 
MDLHydroD due to its implementation of hull emergence angle. The results of this 
study demonstrate good agreement with the MDLHydroD software and provide valuable insights 
into the hydrodynamic forces and motions of vessels at low to moderate forward speeds. However, 
due to time constraints code for the drift force part is not implemented in the HydRA program. 
In the thesis, drift force theory alone is explained.