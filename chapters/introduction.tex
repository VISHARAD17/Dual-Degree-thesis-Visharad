\chapter{INTRODUCTION}

%%%%%%%%%%%%%%%%%%%%% Objective %%%%%%%%%%%%%%%%%%%%%%%%%%%%%%
\section{Objective}
This Dual Degree project has two main objectives. First is the computation of forces and motion characteristics of a floating body with a zero or non-zero speed in regular waves. A 3D panel-based potential theory method using an infinite-depth green function is used for this computation. The second main objective is the computation of second-order forces for drift force. This software can be used for simple single-body cases and complex multi-body cases. Also, the program's output is validated against industry standard software such as Ansys AQWA and WAMIT v6.

The rest of the thesis is organized as follows. Chapter 1, Introduction, Objective of the project, background and motivation, and a literature review of the research paper used for this project are given. Later, in Chapter 2, the mathematical problem setup and its derivation is given. In Chapter 3, the solution to the numerical mathematical problem is shown, and the motions and forces are computed. Similarly, in Chapter 4, Drift force computation is explained. Later in the subsequent chapter results of the project are compared against Softwares like MDLHydroD and WAMIT6.


%%%%%%%%%%%%%%%%% Background and Motivation %%%%%%%%%%%%%%%%%%%%%%
\section{Background and Motivation}
A floating body is subjected to various forces, such as environmental loads due to wind, currents, waves, etc. The sea-keeping problem involves calculating the hydrodynamic parameters and forces acting on a floating structure in waves. These hydrodynamic forces are calculated using numerical methods and potential theory for the wave body interaction problem. This can be further used to calculate the RAOs of various vessels.

Hydrodynamic analysis software can help naval architectures optimize the design of vessels in regular waves. It can provide insights into the vessel's performance in various wave conditions, allowing designers to make informed decisions about the shape and size of the vessel. Also, it saves costs by reducing the need for expensive physical testing.
Software for hydrodynamic analysis can allow designers to iterate designs faster, as they can quickly simulate and evaluate the performance of different design configurations. This can help reduce the time-to-market for new vessels and increase productivity.

The primary goal of this project is to develop a flask-based web application named 'HydRA' capable of computing hydrodynamic properties such as Froude Krylov force, scattering force, RAO, radiation damping 
 , and Added Mass for a vessel in zero and non-zero speed scenarios. Also, the implementation of drift forces captures the effect of second-order forces in the deep sea. The main idea behind building an in-house web application is to have a solid foundation for a hydrodynamic analysis tool that can be further improved and extended toward more complex problems. The main reference theses used for this project are \citet{guha2015estimation}, and \cite{guha2012development}.

%%%%%%%%%%%%%%%%%%%%%% Literature review %%%%%%%%%%%%%%%%%%%%%%%%%%%%%%
\section{Literature review}

In the early stages of ship design, it is crucial to understand the ship's stability to operate efficiently in the sea. Analyzing a ship's hydrodynamic forces and motion characteristics in regular waves is crucial in  designing a stable and safe vessel. With the increase in the availability of computational power, computational fluid dynamics (CFD) has gained popularity in solving fluid-structure interaction problems. However, CFD still takes a significant time to estimate the motion response of floating structures to regular and random waves and is not a viable method during preliminary design. Due to this disadvantage, potential theory-based approaches are still accepted to design floating structures, including offshore oil production platforms, offshore wind turbines, and floating wave energy converters.
The strip theory method, which divides a ship into a number of 2D strips along its length, can be credited with the early developments of the potential theory method. The solutions to various two-dimensional problems are combined to determine the three-dimensional hydrodynamics problem. Numerous authors have contributed to this field, including \cite{newman1979theory}, \cite{ogilvie1969rational}, \cite{beck1990documentation}, \cite{journee2001theoretical}, \cite{salvesen1970ship}. The work of Salvensen, Tuck, and Faltinsen shown in \cite{salvesen1970ship} became the widely accepted method that is still employed by the industry for the construction of floating structures. Since the strip theory relies on the slender body assumption, it was not quite applicable to ship structures and offshore structures, which are non-slender in shape. This led to the development of 3D panel-based potential theory where the floating vessel is discritized into quadrilateral or traingular panels sources of different source strengths.
Tuck and Faltinsen 1970 \cite{salvesen1970ship} provides a detailed review of the theoretical and mathematical models used to predict the motion of ships in waves and the resulting sea loads acting on the hull, which is based on the pressure distribution around the hull and the wave-induced accelerations. The paper covers the effects of ship design parameters, wave steepness and frequency, and wave-induced motions on ship motions. It presents a mathematical model for calculating the sea loads acting on the hull. The paper then discusses the effect of ship design parameters, such as beam and draft, on ship motions and the influence of wave steepness and frequency.
The calculation of the Green function efficiently for frequency and time domain analyses has been a long-standing research area. \cite{newman1979theory} provides a comprehensive list of methods to compute both frequency domain and time domain Green functions efficiently. Similar work on the zero speed frequency domain Green function has also been reported by \cite{telste1986numerical}. It describes the theoretical background of the green function and the numerical method used to evaluate it.

\cite{guha2013development} and \cite{guha2015estimation} discuss developing and validating a frequency domain program that uses numerical methods for predicting the motion and hydrodynamic forces acting on floating bodies, specifically ships.

% something about drift force

This project documents the theory behind developing an in-house wave analysis tool named `HydRA' (Hydrodynamic Response Analysis) that can accurately predict the wave loads and motions of a floating structure in various sea conditions. This tool is based on the potential theory of fluid mechanics and uses a numerical method to solve the wave-body interaction problem. The paper provides a detailed description of the theoretical background, numerical implementation, and tool validation. It demonstrates its effectiveness in predicting the wave loads and motions of different floating structures or ships. An in-house tool is developed in this project is to enable the development further to extend it to analyze complex nonlinear hydrodynamic problems such as maneuvering in waves and parametric excitation of ships and offshore structures. 
