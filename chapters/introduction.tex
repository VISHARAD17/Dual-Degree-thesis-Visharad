\chapter{INTRODUCTION}

%%%%%%%%%%%%%%%%%%%%% Objective %%%%%%%%%%%%%%%%%%%%%%%%%%%%%%
\section{Objective}
This Dual Degree project aims to acheive two primary goals. The first objective is to calculate 
the motion characteristics and forces on a floating structure traveling at a low to moderate 
forward speed in regular waves for the {\bf HydRA}. This computation is done by utilizing a 3D 
panel-based potential theory method that involves using an infinite-depth green function. 
The second main objective is to compute the second-order forces i.e. Mean drift force. 
The implementation is done by keeping in mind that this method can be used for single body case 
as well as for multibody case. Also, then the program's output is validated against 
the MDLHydroD software.

The thesis is structured in the following manner. The first chapter includes an introduction, 
the objectives of the project, the background and motivation, and a literature review of the 
research paper utilized in this project. The second chapter provides an explanation of the 
mathematical problem setup and derivation. Chapter 3 illustrates the solution to the numerical 
mathematical problem. Chapter 4, elaborates on the calculation of wave 
forces and motions. Likewise, Chapter 5 clarifies 
the computation of drift force. In the last chapter, the project outcomes are compared with 
other software tools such as MDLHydroD and Ansys AQWA.


%%%%%%%%%%%%%%%%% Background and Motivation %%%%%%%%%%%%%%%%%%%%%%
\section{Background and Motivation}
A floating object is exposed to multiple forces, such as wind, currents, waves, and other 
environmental loads. The issue of sea-keeping involves the computation of hydrodynamic 
parameters and forces acting on a floating structure when it is in waves. To solve the 
problem of wave-body interaction, hydrodynamic forces are calculated using numerical 
methods and potential theory. These calculations can then be used to compute the RAOs 
of different types of vessels.

Hydrodynamic analysis software can help naval architectures optimize and enhance the design of 
vessels in regular waves. It can provide insights into the vessel's performance in various 
wave conditions, 
allowing designers to make informed decisions about the shape and size of the vessel. 
Moreover, it saves costs by reducing the need for expensive physical testing.
Software for hydrodynamic analysis can allow designers to iterate designs faster, 
as they can quickly simulate and evaluate the performance of different design configurations. 
This can help reduce the time-to-market for new vessels and increase productivity.

The primary goal of this project is to develop a flask-based web application named 'HydRA' 
capable of computing hydrodynamic properties such as Froude Krylov force, scattering force, 
RAO, radiation damping, and Added Mass for a vessel in zero and non-zero speed scenarios.
Also, the implementation of drift forces captures the effect of second-order forces in 
the deep sea. The main idea behind building an in-house web application is to have a 
solid foundation for a hydrodynamic analysis tool that can be further improved and 
extended toward more complex problems. The main reference theses used for this project 
are \citet{guha2015estimation}, and \cite{guha2012development}.

%%%%%%%%%%%%%%%%%%%%%% Literature review %%%%%%%%%%%%%%%%%%%%%%%%%%%%%%
\section{Literature review}

During the initial phases of ship design, it is vital to have knowledge about the vessel's stability to ensure efficient and safe 
operation at sea. Evaluating hydrodynamic forces and motion features of the ship in regular 
waves is an essential factor for designing a stable and secure vessel. As computational power 
has become more accessible, computational fluid dynamics (CFD) has become more popular 
in solving problems related to the interaction between fluids and structures. 
Despite advancements in CFD, it is a still time consuming process to predict responses. Therefore, 
CFD may not be a practical option during the preliminary design stage. 
Since, CFD is not feasible during the early stages of design due to its time-consuming nature,
potential theory based methods are still widely used for designing floating structures like oil 
production platforms, offshore wind turbines, wave energy converters, etc.
The potential theory method has been advanced by the strip theory approach, 
which involves dividing a ship into multiple 2D strips along its length. 
By combining solutions to various two-dimensional problems, the three-dimensional 
hydrodynamics problem can be determined. Many researchers have contributed to the 
development of this approach which includes \cite{newman1979theory}, 
\cite{ogilvie1969rational}, \cite{beck1990documentation}, \cite{journee2001theoretical}, 
\cite{salvesen1970ship}. Beacause, the strip theory method assumes the bodies are slender, it was 
not suitable for analyzing ship structures and offshore stuctures which have a non-slender shape.
Therefore, a 3D panel based potential theory was developed. In this method, the floating vessel is
descritized or is a mesh of quadrilateral or triangular panels with different source strengths.

Tuck and Faltinsen 1970 \cite{salvesen1970ship} provides a detailed review of the theoretical 
and mathematical models used to predict the motion of ships in waves and the resulting sea 
loads acting on the hull, which is based on the pressure distribution around the hull and 
the wave-induced accelerations. The paper covers the effects of ship design parameters, 
wave steepness and frequency, and wave-induced motions on ship motions. It presents a 
mathematical model for calculating the sea loads acting on the hull. The paper then 
discusses the effect of ship design parameters, such as beam and draft, on ship motions 
and the influence of wave steepness and frequency.

The calculation of the Green function efficiently for frequency and time domain analyses has 
been a long-standing research area. \cite{newman1979theory} provides a comprehensive list of 
methods to compute both frequency domain and time domain Green functions efficiently. 
Similar work on the zero speed frequency domain Green function has also been reported 
by \cite{telste1986numerical}. It describes the theoretical background of the green 
function and the numerical method used to evaluate it.

The book \cite{liapis1986time} is considered very important book for studying dynamics of ships and
floating structures. The book focuses on the time-domain analysis of ship motions and provides 
an in-depth discussion of various methods for solving the differential equations of motion 
for ships in time-domain, including numerical and analytical methods. This book also explains about the
equations shown in the section \ref{sec:numerical_dis}.

\cite{guha2013development} and \cite{guha2015estimation} discuss developing and validating a 
frequency domain program that uses numerical methods for predicting the motion and hydrodynamic 
forces acting on floating bodies, specifically ships.

When a floating body is subjected to ocean waves, it experiences not only first-order forces 
(linear wave forces) but also second-order forces, which are nonlinear in nature. 
The second order forces are better known by their physical effects on a floating
body as the added resistance or the mean drift forces
Second order forces become important when wave induced motions are large. 
Primarily, mean drift forces are computed using two menthods:
(a) far feild method and (b) near field method. 

Far field method was introduced by \cite{maruo1957excess} which is based on diffracted and radiated
wave energy and momentum flux at infinity.
The paper analyzes the experimental 
data of model tests and the calculations based on hydrodynamic theory to estimate the additional 
resistance or mean drift force on the ship due to wave-induced effects. The study indicates that the additional 
resistance is proportional to the square of the wave amplitude and frequency, which is 
consistent with the second-order wave theory.

Near feild method was introduced by \cite{boese1970einfache} which is based on direct integration of 
pressure on submerged surface inshort uses potential theory.
This paper, describes a simple method for calculating the increase in resistance of a ship in waves. 
The method involves calculating the wave drag of the ship at zero speed and then using a 
correction factor to account for the effect of waves on the ship's resistance.

\cite{faltinsen1980prediction},  \cite{pinkster1980low} talks about the derivation of the equations 
for Mean drift forces. And also about the second order wave theroy. 
\cite{faltinsen1980prediction} presents a method to predict the added resistance or drift force 
and propulsion of a ship in a seaway. 
The method is based on the potential theory and takes into account the 
effects of wave-induced motions on the ship's performance. The paper provides a detailed 
description of the theoretical background and numerical implementation of the method using
the perturbation theory. 
Faltinsen also presents experimental results that validate the accuracy of the method. 
This paper is a significant contribution to the field of ship hydrodynamics, providing a 
reliable and practical approach to predicting ship performance in a seaway.
\cite{pinkster1980low} concludes that second-order wave forces can have a significant
effect on the motions and loads of floating structures and should be considered in the 
design of offshore structure.

This project documents the development of added feature i.e. to able to perform analysis 
with the effect of forward speed for simple single body case as well as for complex multibody
case for the "HydRA" tool begin developed in the Reaserch guild group at IIT Madras. In addition to 
that theory about drift force and its solution is also disccued. 