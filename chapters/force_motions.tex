\chapter{FORCES and MOTIONS}
\section{Wave Potentials}
Once the $[\alpha]$ matrix is computed using the expressions derived in the previous sections. Using
the eq.(\ref{eq:sigma_alpha}) we can solve for unknown source strength $\sigma$. Here, normal velocity
vector $\{v_n\}$ matrix will be different for radiation and scattering potentials. It is computed by applying 
the boundary conditions given in the section(\ref{sec:Boundary condition}). For $\{v_n\}$ related to
scattering problem requires incident potential given by 
\begin{equation}
    \label{eq:incident_pot}
    \phi_I = \frac{igA}{\omega_I} \exp[-ik_I(x\cos \beta + y\sin \beta)+kz]
\end{equation}
Note that, all the potentials are computed at the centroid of the panels. After applying respective 
boundary conditions for radiation and scattering problem unknown source strength can be calculated for the 
respective potentials / problems.

Once, the $[\beta]$ matrix is computed using the green function explained in sec.(\ref{sec:green_fun}), combining 
it with respective source strength of radiation and scattering problems,using the eq.(\ref{eq:vel_pot})
respective potentials can be computed.

$\phi_D$ is used to denote scattering potential and $\phi_k (k=1, 2, \cdots, 6)$ is used to denote the 
radiation potential in all 6-degrees of freedom. According to \cite{salvesen1970ship} for radiation potential $\phi_k$ 
correction terms are added for $5^{th}$ and $6^{th}$ mode of motion.
\begin{align}
    \phi_j &= \phi^0_j \quad \text{for} j = 1, 2, 3, 4 \\ \nonumber 
    \phi_5 &= \phi_5 +\frac{U}{i\omega_e} \phi^0_3 \\ \nonumber 
    \phi_6 &= \phi^0_6 - \frac{U}{i\omega_e} \phi^0_2
\end{align}

$\phi^0_j$ are the zero speed terms computed without the effect forward speed, process to 
compute these quantities is given in \cite{guha2012development}. But, there is a slight change while
computing the zero speed terms given in this paper which is that in the boundary condition used,
forward speed is considered and the boundary condition given in \ref{sec:Boundary condition} is used.
Then after, to the obtained radiation potentials, perticulary for $5^{th}$ and $6^{th}$ mode of motions 
extra terms are added according to the above equations. Once, the radiation potential is computed 
added mass and radiation damping can be obtained.
\section{Exciting forces}
Once all potentials are computed, through direct integration on the submerged surface forces 
can be computed.
{\bf Froude Krylov force} is calculated using incident wave potential given in eq.(\ref{eq:incident_pot}). Hence, 
equation for Froude Krylov force / incident force for modes $(k=1, 2, \cdots, 6)$ is 
\begin{equation}
    F_I^k = -\iota \omega\int_{S} \phi_I n_k ds
\end{equation}
Similar to Froude Krylov force, {\bf Scattering force} is obtained through integration of scattering force over the 
submerged surface. Hence, for all modes $(k=1, 2, \cdots, 6)$ equation for scattering force is 
\begin{align}
    F_D^k &= \rho \int_{S}(\iota \omega_e n_k - Um_k)\phi_D\,ds \\ \nonumber
    &= -\rho \int_{S}\phi_k^0\frac{\partial \phi_I}{\partial n} ds \quad \text{for} k=1, 2, 3, 4 \\ \nonumber 
    &= -\rho \int_{S}\phi_k^0\frac{\partial \phi_I} ds + \frac{\rho U}{i\omega_e}\int_{S}\phi^0_3 
    \frac{\partial \phi_I} ds \quad \text{for} k=5 \\ \nonumber
    &= -\rho \int_{S}\phi_k^0\frac{\partial \phi_I} ds - \frac{\rho U}{i\omega_e}\int_{S}\phi^0_2
    \frac{\partial \phi_I} ds \quad \text{for} k=6
\end{align}
Now, the total {\bf exciting force} amplitute (Fourde Krylov + Scattering) in mode $(k=1, 2, \cdots 6)$ is given by 
\begin{equation}
    \boldsymbol{F} = F^k_I + F^k_D 
\end{equation}
\section{Radiation forces}
The radiation wave load/force for modes $(k=1, 2, \cdots, 6)$ is given by 
\begin{align}
    F_R^k &= -\iota \omega_e \rho \int_{S}\phi_k n_k ds \\ \nonumber
          &= \omega_e^2 A_{jk} - \iota \omega_e B_{jk}
\end{align}
In the expansion of above equation there are two terms, the term proportional to the acceleration is 
added mass and the one proportional to the velocity is radiation damping. Equations for added mass and
radiation damping is given as 
\begin{align}
    A_{jk} &= -\frac{\rho}{\omega_e}\int_{S}\text{Im}(\phi_k) n_j ds \\ 
    B_{jk} &= -\rho \int_{S} \text{Re}(\phi_k)n_j ds 
\end{align}
Note, that Added mass $A_{jk}$ and Radiation damping $B_{jk}$ are real valued. Also the subscript $(jk)$ denotes
the added mass or radiation damping in $j^{th}$ mode of motion due to the body oscillation along the 
$k^{th}$ direction.
\subsection{Forward speed RAO}
For a regular incident wave of unit amplitute the steady state motion will also sinusoidal and the steady 
state amplitute $\{\xi\}$ is governed by :
\begin{align}
    -\omega^2[M + A]\{\xi\} + i\omega[B]\{\xi\} + [C]\{\xi\} = \{F\}
\end{align}

where $[M]$ represents the mass matrix, $[A]$ represents the added mass matrix, $[B]$ denotes the radiation damping matrix, $[C]$ denotes the stiffness matrix, $\{F\}$ denotes the external force vector and $\{\xi\}$ denotes the vector of steady state displacement amplitude. The response amplitude operator $\{\xi\}$ is a $6N_b \times 1$ vector given by

\begin{align}
    \{\xi\} = \left(-\omega^2[M + A] + i\omega[B] + [C]\right)^{-1} \{F\}
\end{align}
