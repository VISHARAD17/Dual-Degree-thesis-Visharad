\chapter{NUMERICAL SOLUTION}
The complex potential on the submerged surface of the vessel is the key to solving the hydrodynamic problem. 
Once the potential is known, all the hydrodynamic coefficients required 
to obtain all the vessel response, i.e., added mass, damping, excitation forces and 
response ampliture operator (RAO) can be calculated.

The boundary value problem described in (\ref{eq:kin_free_surface_cond}) to 
(\ref{eq:radiation_boundary})  will be solved by a boundary element method. 
The body surface will be a suitable distribution of source singularities that satisfy 
the governing equation and the boundary conditions. Solving a Fredholm integral of the 
second kind over the discretized body boundary will determine the source strengths. 

%%%%%%%%%%%%%%%%%%%%%%%%%%%%%%%%%%%%%%%%%%%%%%%%%%%%%%%%%%%%%%%%%%%%%%%%%%%%%%%%%%%%%%%%%%%%%%
\section{Integral equation}
The velocity potential at some point $(x, y, z)$ in the fluid domain can be expressed in 
terms of the surface distribution of sources.

\begin{equation}
    \label{eq:vel_pot}
    \phi(\vec{x}) = \frac{1}{4\pi}\int_S \sigma(\vec{x_s})G(\vec{x}:\vec{x_s})\,dS
\end{equation}

where $\vec{x_s}$ denotes the source point of the body surface, $(\vec{x})$ denotes the 
point where the potential is being calculated, and $\sigma(\vec{x_s})$ is the unknown 
source strength distribution on the source point. Section (\ref{sec:green_fun})
explains about the Green function. 
The Green function satisfies the continuity condition and all the boundary conditions, 
including the free surface and radiation boundary condition, except for the following 
normal velocity boundary condition on the hull surface, which is:

\begin{equation}
    \label{eq:vel_boundary_cond}
    \frac{\partial \phi}{\partial n} = v_n \quad \text{on}\; S
\end{equation}

Here, $v_n(x, y, x)$ is the flow normal velocity on the hull surface, after applying the 
velocity boundary condition given in (\ref{eq:velocity_potential}) on the velocity 
potential (\ref{eq:vel_pot}), the below equation is obtained, also known as Fredholm integral 
of the second kind :
\begin{equation}
    \label{eq:integral_eq}
    -\frac{1}{2}\sigma(\vec{x_s}) + \frac{1}{4\pi}\int_S\sigma(\vec{x_s})\frac{\partial G(\vec{x}:\vec{x_s})}{\partial n} = v_n \quad \text{on}\; S
\end{equation}
Here, $\frac{\partial G}{\partial n}$ denotes the derivative of the Green function in the outward normal direction. This normal derivative of $G$ can be 
evaluated using its partial dervitives in $x$, $y$ and $z$ as :
\begin{equation}
    \frac{\partial G}{\partial n} = \frac{\partial G}{\partial x}n_x + \frac{\partial G}{\partial y}n_y + \frac{\partial G}{\partial z}n_z
\end{equation}
% The oscilatory potential components $\phi_k(k=1, 2, \cdots, 6)$ must satisfy the free-surface condition
% and the condition that the fluid on the hull surface must move identically to the hull surface. Hence, 
% \begin{equation}
%     \frac{\partial \phi_j}{\partial n} = i\omega_e n_j + Um_j \quad \text{on}\; S
% \end{equation}
Now, while calculating influence on a panel because of other source panel, the panel where the 
potential is being calculated and the source panel becomes the same. Due to this, frequency 
independant part of the green function tends to go to infinity. This case is known as 
singularity. To compensate for that influence, $-\frac{1}{2}\sigma$ is added into the 
equation according to the residual theory.



\section{Numerical Discretization}
\label{sec:numerical_dis}
To solve (\ref{eq:integral_eq}), the body surface is discretized into $M$ panels of area 
$\Delta S_m$ where $m = (1, 2, \cdots, M)$. Every entity presented here is computed at the centroid of the panel.
Hence, the source strength as well is computed over each panels centroid. Now, for $i^{th}$ panel 
(\ref{eq:integral_eq}) can be written as:
\begin{equation}
    \label{eq:dis_integral_eq}
    -\frac{1}{2}\sigma_i(\vec{x_s}) + \frac{1}{4\pi}\int_S\sigma_i(\vec{x_s})\frac{\partial G_i(\vec{x}:\vec{x_s})}{\partial n} = v_{n_i} \quad \text{on}\; S
\end{equation}
Here, the subscript $i$ in the equation represents the entities of an individual panel.

The equation (\ref{eq:dis_integral_eq}) can be re-written as:
\begin{equation}
    \label{eq:alpha_int_eq}
    -\frac{1}{2}\sigma_i + \frac{1}{2}\sum_{i=1, j=1, i\ne j}^{M}\alpha_{ij}\sigma_i = v_{n_i} \quad \text{for} i = 1, 2, \cdots M
\end{equation}
Where, $\alpha_{ij}$ is : 
\begin{equation}
    \label{eq:alpha}
    \alpha_{ij} = \frac{1}{2\pi}\int_{\Delta S_i}\frac{\partial G(x, x_s)}{\partial n} \,dS
\end{equation}
In physical terms, $\alpha_{ij}$ denotes the velocity induced at the $i^{th}$ panel in the direction normal to the surface by a source distribution of unit strength distributed uniformaly over the $j^{th}$ panel which is the source panel here and $i^{th}$ panel is the influenced panel where the potential is being calculated because of the presence of other source panels.

The equation (\ref{eq:alpha_int_eq}) can be written in the matrix format as :
\begin{equation}
    \label{eq:sigma_alpha}
    [\sigma] = 2[\alpha - I]^{-1}[v_n]
\end{equation}
where, $I$ represents the unit matrix. $[\alpha]$ is an $M\times M$ cross matrix where $M$ is 
the total number of panels. Similarly, (\ref{eq:vel_pot}) can be reduced to : 
\begin{equation}
    \label{eq:beta_eq}
    \phi_i = \sum_{j=1}^{M}\beta_{ij}\sigma_{j}
\end{equation}
where, $[\beta]$ is the $M\times M$ matrix given by :
\begin{equation}
    \label{eq:beta_eq}
    \beta_{ij} = \frac{1}{4\pi}\int_{\Delta S_j}G(\Vec{x_i}, \vec{x_s}) dS  
\end{equation}
Now, the equation \ref{eq:beta_eq} can be written in the matrix form as :
\begin{equation}
    \{\phi\} = [\beta]\{\sigma\}
\end{equation}

Threfore, the boundary problem can be solved for the velocity potential on the body surface 
if the value of the integral of the Green function $G$ and its derivative which is
$\frac{\partial G}{\partial n}$ are known.
%%%%%%%%%%%%%%%%%%%%%%%%%%%%%%%%%%%%%%%%%%%%%%%%%%%%%%%%%%%%%%%%%%%%%%%%%%%%%%%%%%%%%%%%%%%%
\section{Green's Function}
\label{sec:green_fun}
For the calculation of radiation and diffraction potential, an infinite-depth green function is 
used. This Green function and its numerical solution is explained in \cite{telste1986numerical}. 
This green function assumes source distribution over the submerged surface of the vessel. 
It relates the unknown source strength to the velocity potential. Here, the green function 
comprises two parts: one is the frequency-dependent part, and the other is the 
frequency-dependent part.  The Independent part represents the simple potential 
and interaction between the body surface and the free surface. On the other hand, 
the frequency dependent part is to comprise the oscillating potential due to the 
oscillating source. The equation of the Green function is given as 
\begin{equation}
    \label{eq:green_eq}
    G(\vec{x}, \vec{x_s}, \omega_e) = \frac{1}{r} + \frac{1}{\acute{r}} + \Tilde{G}(\vec{x}, \vec{x_s}, \omega_e) 
\end{equation}
where, 
\begin{equation}
    r = ||\boldsymbol{x} - \boldsymbol{x_s}|| = \sqrt{(x-x_s)^2+(y-y_s)^2+(z-z_s)^2}
\end{equation}
represents the Eucledian distance between the field point $P(x, y, z)$ and source 
point $Q(x_s, y_s, z_s)$. The frequency used here is the encounter frequency shown in 
(\ref{eq:omega}).

\begin{equation}
    r' = \sqrt{(x - \xi)^2 + (y - \eta)^2 + (z + \zeta)^2}   
\end{equation}
represents the Euclidean distance between the field point $P(x, y, z)$ and point 
$Q'(\xi, \eta, \zeta)$, which is the image of the source point in the waterplane 
$z=0$ and $\tilde{G}(\vec{x}, \vec{x_s}, \omega_e)$ represents the frequency domain 
part of the green function, also refree.

Now, substituting (\ref{eq:green_eq}) in (\ref{eq:alpha}) and (\ref{eq:beta_eq}), 
gives the following equations for $\alpha$ and $\beta$ matrices :
\begin{align}
    \label{eq:alpha}
    \alpha_{ij} &= -\frac{1}{2\pi}\iint_{\Delta S_i} \frac{\partial}{\partial n}\left(\frac{1}{r}\right) dS 
    - \frac{1}{2\pi}\iint_{\Delta S_j} \frac{\partial}{\partial n}\left(\frac{1}{r'}\right) dS \\ \nonumber
    &- \frac{1}{2\pi}\iint_{\Delta S_j} \frac{\partial G(\vec{x}_i, \vec{x}_j, \omega_e) }{\partial n}\,dS
\end{align}
\begin{align}
    \label{eq:beta}
    \beta_{ij} &= -\frac{1}{4\pi}\iint_{\Delta S_j}\left(\frac{1}{r}\right) dS 
    - \frac{1}{4\pi}\iint_{\Delta S_j}\left(\frac{1}{r'}\right) dS \\ \nonumber
    &-\frac{1}{4\pi}\iint_{\Delta S_j} \Tilde{G}(x_i, x_j, \omega_e)dS
\end{align}
Here, $j^{th}$ panel is the source panel and $i^{th}$ panel is the feild point where 
potential is being calculated. In the above expressions, first two terms are 
frequency independant and the last term is frequency dependent which can be obtained by
solving for the wavy green function.
%%%%%%%%%%%%%%%%%%%%%%%%%%%%%%%%%%%%%%%%%%%%%%%%%%%%%%%%%%%%%%%%%%%%%%%%%%%%%%%%%%%%

\section{Frequency independent part of Green function}
The frequency independant part of the $[\alpha]$ matrix can be computed analytically using 
the expressions provided by \cite{hess1964calculation}. 
Similarly, the frequency independant terms of $[\beta]$ matrix can be evaluated using the 
expressions given by \cite{katz2001low}. Normal derivatives can be expressed as sum of 
dot products of normal unit vectors and the partial derivatives in $x$, $y$ 
and $z$ directions. Hence, 
\begin{align}
    \label{eq:one_by_r_comp}
    \iint_{\Delta S_i} \frac{\partial}{\partial n}\left(\frac{1}{r}\right) dS &= 
    n_x \iint_{\Delta S_j} \frac{\partial}{\partial x}\left(\frac{1}{r}\right) dS + 
    n_y \iint_{\Delta S_j} \frac{\partial}{\partial y}\left(\frac{1}{r}\right) dS \\ \nonumber 
    &+ n_z \iint_{\Delta S_j}\frac{\partial}{\partial z}\left(\frac{1}{r}\right) dS
\end{align}
Above integration is performed over the source panel $\Delta S_j$ and $n_x$, $n_y$, $n_z$ are the 
unit normal components of the feild panels. The integral is evaluated in local co-ordinate system 
with its origin at centroid. The coordinates of the source panel's 4 vertices are 
$A(x_1, y_1, 0)$, $B(x_2, y_2, 0)$, $C(x_3, y_3, 0)$ and $D(x_4, y_4, 0)$, which are in BCS coordinate system. 
Here, only $x$ and $y$ co-ordinates of the panel vertices are considered.Hence, according 
to \cite{hess1964calculation} equations for the partial derivatives 
required to compute the normal derivative (\ref{eq:one_by_r_comp}) are given below :

\begin{align}
    \begin{split}
        \label{eq:partial_r_x}
        \iint_{\Delta S_j} \frac{\partial}{\partial x}\left(\frac{1}{r}\right) dS &=\frac{y_{2}-y_{1}}{d_{12}} \ln \left(\frac{r_{1}+r_{2}-d_{12}}{r_{1}+r_{2}+d_{12}}\right)\\ 
        &+ \frac{y_{3}-y_{2}}{d_{23}} \ln \left(\frac{r_{2}+r_{3}-d_{23}}{r_{2}+r_{3}+d_{23}}\right)\\ &+\frac{y_{4}-y_{3}}{d_{34}} \ln \left(\frac{r_{3}+r_{4}-d_{34}}{r_{3}+r_{4}+d_{34}}\right)\\
        &+\frac{y_{1}-y_{4}}{d_{41}} \ln \left(\frac{r_{4}+r_{1}-d_{41}}{r_{4}+r_{1}+d_{41}}\right)
    \end{split}
\end{align}

\begin{align}
    \begin{split}
        \label{eq:partial_r_y}
        \iint_{\Delta S_j} \frac{\partial}{\partial y}\left(\frac{1}{r}\right) dS
        &= \frac{x_{1}-x_{2}}{d_{12}} \ln \left(\frac{r_{1}+r_{2}-d_{12}}{r_{1}+r_{2}+d_{12}}\right)\\ &+\frac{x_{2}-x_{3}}{d_{23}} \ln \left(\frac{r_{2}+r_{3}-d_{23}}{r_{2}+r_{3}+d_{23}}\right) \\
        &+\frac{x_{3}-x_{4}}{d_{34}} \ln \left(\frac{r_{3}+r_{4}-d_{34}}{r_{3}+r_{4}+d_{34}}\right)\\ &+\frac{x_{4}-x_{1}}{d_{41}} \ln \left(\frac{r_{4}+r_{1}-d_{41}}{r_{4}+r_{1}+d_{41}}\right)
    \end{split}
\end{align}

\begin{align}
    \begin{split}
        \label{eq:partial_r_z}
        \iint_{\Delta S_j} \frac{\partial}{\partial z}\left(\frac{1}{r}\right) dS 
        &=\tan ^{-1}\left(\frac{m_{12} e_{1}-h_{1}}{z r_{1}}\right)-\tan ^{-1}\left(\frac{m_{12} e_{2}-h_{2}}{z r_{2}}\right) \\
        &+\tan ^{-1}\left(\frac{m_{23} e_{2}-h_{2}}{z r_{2}}\right)-\tan ^{-1}\left(\frac{m_{23} e_{3}-h_{3}}{z r_{3}}\right) \\
        &+\tan ^{-1}\left(\frac{m_{34} e_{3}-h_{3}}{z r_{3}}\right)-\tan ^{-1}\left(\frac{m_{34} e_{4}-h_{4}}{z r_{4}}\right) \\
        &+\tan ^{-1}\left(\frac{m_{41} e_{4}-h_{4}}{z r_{4}}\right)-\tan ^{-1}\left(\frac{m_{41} e_{4}-h_{1}}{z r_{1}}\right)
    \end{split}
\end{align}
\newpage
Similarly, according to \cite{katz2001low} the frequency dependent integrals of the $[\beta]$ matrix
 (\ref{eq:beta}) are given by the following expression :
\begin{align}
    \begin{split}
        \label{eq:frq_indi_beta}
       \iint_{\Delta S_j}{\left(\frac{1}{r}\right)} dS &= \left[\frac{(x-x_1)(y_2-y_1) - (y-y_1)(x_2-x_1)}{d_{12}}\ln\left(\frac{r_1+r_2+d_{12}}{r_1+r_2-d_{12}}\right)\right. \\
&+ \left.\frac{(x-x_2)(y_3-y_2)-(y-y_2)(x_3-x_2)}{d_{23}}\ln\left(\frac{r_2+r_3+d_{23}}{r_2+r_3-d_{23}}\right)\right. \\
&+ \left.\frac{(x-x_3)(y_4-y_3) -(y-y_4)(x_4-x_3)}{d_{34}}\ln\left(\frac{r_3+r_4+d_{34}}{r_3+r_4-d_{34}}\right)\right. \\
&+ \left.\frac{(x-x_4)(y_1-y_4) -(y-y_4)(x_1-x_4)}{d_{41}}\ln\left(\frac{r_4+r_1+d_{41}}{r_4+r_1-d_{41}}\right)\right] \\
&- z\left[\tan^{-1}\left(\frac{m_{12}e_1-h_1}{zr_1}\right)-\tan^{-1}\left(\frac{m_{12}e_2-h_2}{zr_2}\right)\right. \\
&+ \left.\tan^{-1}\left(\frac{m_{23}e_2-h_2}{zr_2}\right)-\tan^{-1}\left(\frac{m_{23}e_3-h_3}{zr_3}\right)\right.\\
&+ \left.\tan^{-1}\left(\frac{m_{34}e_3-h_3}{zr_3}\right)-\tan^{-1}\left(\frac{m_{34}e_4-h_4}{zr_4}\right)\right.\\
&+ \left.\tan^{-1}\left(\frac{m_{41}e_4-h_4}{zr_4}\right)-\tan^{-1}\left(\frac{m_{41}e_1-h_1}{zr_1}\right)\right] 
    \end{split}
\end{align}
In the above expression, instead of $|z|$ as reported in \cite{katz2001low}, only $z$ is used. This modifies expression is found 
to be in good agree with the point source approximation as the distance between the field panel and source panel increases.

In the expressions (\ref{eq:partial_r_x}), (\ref{eq:partial_r_y}), (\ref{eq:partial_r_z}) and (\ref{eq:frq_indi_beta}) the intermediate terms are given by the below equations:
\begin{align}
    d_{12} &= \sqrt{(x_2-x_1)^2+(y_2-y_1)^2}\\
    d_{23} &= \sqrt{(x_3-x_2)^2+(y_3-y_2)^2}\\
    d_{34} &= \sqrt{(x_4-x_3)^2+(y_4-y_3)^2}\\
    d_{41} &= \sqrt{(x_1-x_4)^2+(y_1-y_4)^2}
\end{align}
\newpage
\noindent\begin{minipage}{.5\linewidth}
    \begin{equation}
        m_{12} =\frac{y_2-y_1}{x_2-x_1}
    \end{equation}
    \end{minipage}%
    \begin{minipage}{.5\linewidth}
    \begin{equation}
        m_{23} =\frac{y_3-y_2}{x_3-x_2}
    \end{equation}
\end{minipage}
\noindent\begin{minipage}{.5\linewidth}
    \begin{equation}
        m_{34} =\frac{y_4-y_3}{x_4-x_3}
    \end{equation}
    \end{minipage}%
    \begin{minipage}{.5\linewidth}
    \begin{equation}
        m_{41} =\frac{y_1-y_4}{x_1-x_4}
    \end{equation}
\end{minipage}
\begin{align}
    r_1 &= \sqrt{(x-x_1)^2+(y-y_1)^2+z^2}\\
    r_2 &= \sqrt{(x-x_2)^2+(y-y_2)^2+z^2}\\
    r_3 &= \sqrt{(x-x_3)^2+(y-y_3)^2+z^2}\\
    r_4 &= \sqrt{(x-x_4)^2+(y-y_4)^2+z^2}
\end{align}
\begin{align}
    e_1 &= z^2+(x-x_1)^2\\
    e_2 &= z^2+(x-x_2)^2\\
    e_3 &= z^2+(x-x_3)^2\\
    e_4 &= z^2+(x-x_4)^2
\end{align}

\section{Frequency dependent part of Green function}
The frequency dependent part of the green function $\Tilde{G}(x, x_s, \omega_e)$ 
and its derivatives are evaluated using the numerical methods given in 
\cite{telste1986numerical}. The given expressions for derivatives and integrals 
are shown below : 
\begin{align}
    \tilde{G}(\boldsymbol{x},\boldsymbol{x_s}, \omega_e) &= 2f\left[ R_0(h,v) - i \pi J_0(h)e^{v}\right]\\
    \frac{\partial{\tilde{G}_0}}{\partial{\rho_G}} &= -2f^{2}\left[R_1(h,v)-i\pi J_{1}(h)e^v\right]\\
    \frac{\partial{\tilde{G}_0}}{\partial{x}} &=\frac{(x^e_P-x^e_Q)}{\rho_G}\frac{\partial{\tilde{G}_0}}{\partial{\rho_G}}\\
    \frac{\partial \tilde{G}_{0}}{\partial y} &= \frac{\left(y^e_P-y^e_Q\right)}{\rho_G} \frac{\partial \tilde{G}_{0}}{\partial \rho_G}\\
    \frac{\partial \tilde{G}_{0}}{\partial z} &= 2 f^{2}\left[\frac{1}{d}+R_{0}(h, v)-i \pi J_{0}(h) e^{v}\right]
\end{align}
where, 
\begin{align}
    f &= \frac{\omega^2 L}{g} \\
    \rho_G &= \left[(x^e_P-x^e_Q)^2 + (y^e_P-y^e_Q)^2\right]^{\frac{1}{2}} \\
    r &= \left[\rho^2_G+(z^e_P-z^e_Q)^2\right]^{\frac{1}{2}} \\
    {r}^{\prime}&= \left[\rho^2_G+(z^e_P+z^e_Q)^2\right]^{\frac{1}{2}} \\
    h &= f\rho_G \\
    v &= f(z^e_P+z^e_Q) \\
    d &= f {r}^{\prime}
\end{align}

where, $J_0$ and $J_1$ are the first-kind Bessel functions, $L$ is a non-dimensionalizing length 
selected by the user, 
and $g$ is the acceleration due to gravity. In this thesis $L=1$ is used. 
Real valued functions $R_0(h,v)$ and $R_1(h,v)$ are effectively evaluated 
using the numerical method suggested by \cite{telste1986numerical}.

Once, the potentials are computed pressure can be obtained on each panel.
Through integration of pressure throughout the submerged surface forces can be evaluated. 
Using radiation potentials added mass and radiation damping can be computed. 
After, obtaining all these coeffecients and using them motion equation can be solved to get the RAOs.
Steps to compute these forces and motion are explained in detail in the next section.